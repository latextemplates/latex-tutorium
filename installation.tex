%       $Id: installation.tex,v 1.17 2005/06/28 18:30:35 bronger Exp $    
%
%     installation.tex -- Part of the LaTeX Tutorium
%     Copyright 2004 Project Members of
%                    http://sourceforge.net/projects/latex-tutorium/
%                    
%
%   This program is free software; you can redistribute it and/or
%   modify it under the terms of the Artistic License 2.0 as published
%   by Larry Wall.  You should have received a copy of the Artistic
%   License 2.0 along with this program in the file COPYING; if not,
%   you can get it at
%     http://dev.perl.org/rfc/346.html
%   or contact the current maintainers of the LaTeX Tutorium.
%
%   This program is distributed in the hope that it will be useful, but
%   WITHOUT ANY WARRANTY; without even the implied warranty of
%   MERCHANTABILITY or FITNESS FOR A PARTICULAR PURPOSE.  See the
%   Artistic License 2.0 for more details.
%
%   This file may only be distributed together with a copy of the LaTeX
%   Turorium.
%
%   The LaTeX Tutorium consists of all files listed in manifest.txt.


\section{Die Installation unter Windows}
\index{Installation}
\index{Download}
\index{CD@\CD}

Es gibt mehrere M�glichkeiten, an \LaTeX{} heranzukommen:
\begin{itemize}
\item Man nutzt eine sehr gute Internet-Verbindung, um es sich herunterzuladen.
  Modem kann man vergessen, mit \versalien{ISDN} ist etwas Geduld n�tig.  Ab
  \versalien{DSL} aufw�rts geht es mehr oder minder flott.
\item Man kauft sich eine \versalien{CD}\@.  Die gibt es f�r wenig Geld in
  einigen B�chereien (z.\,B. Lehmanns), man kann sie aber auch online
  bestellen.  Auf der \versalien{CD} ist u.\,U. nicht exakt das drauf, was hier
  im Tutorium verwendet wird.
\item Man kennt einen netten Menschen, der's hat.
\end{itemize}

\bigskip
\noindent
Von den folgenden vier Programmen m�ssen drei installiert sein, um mit \LaTeX{}
arbeiten zu k�nnen.  FreePDF ist nicht unbedingt n�tig.


\begin{table}
  \caption{Die Komponenten von \LaTeX{} zum Herunterladen.  Sie sollten in
      genau dieser Reihenfolge installiert werden.}
  \label{tab:Downloads}
  \vspace{1ex}
  \begin{tabular}{@{}llc@{}}
    \toprule
    Komponente       &  Homepage                 &  Gr��e          \\
                     &                           &  (Megabytes)    \\
    \midrule
    Acrobat Reader   &  \url{www.adobe.de}       &  $18$           \\
    MiK\TeX{}        &  \url{www.miktex.org}     &  $24$           \\
    FreePDF (optional)
                     &  \url{freepdfxp.de/fpxp.htm}&  $11$         \\
    \TeXnicCenter{}  &  \url{www.toolscenter.org}  &  $4$          \\
    \bottomrule
  \end{tabular}
\end{table}

\subsection{Acrobat Reader}
\index{Reader, Acrobat Reader}
\index{Acrobat Reader}
\index{Adobe}
\index{PDF@\PDF}

Mit ein bisschen Gl�ck ist der Acrobat Reader bereits auf dem Computer
vorhanden, weil es bei einigen anderen Programmen dabei ist.  Wenn nicht, muss
man es sich bei `\url{www.adobe.de}' herunterladen, das sind knapp 18\,MB\@.

Die Datei, die man dabei erh�lt, ruft man auf, und das ergibt dann alles
weitere.  Die Installation selber dauert nur zwei Minuten.


\subsection{\LaTeX}
\index{MiKTeX@MiK\TeX}

Nat�rlich braucht man \LaTeX{} selber.  Das bekommt man unter
`\url{www.miktex.org}' in Form vom sogenannten "`MiK\TeX"'.  Man kann es
entweder herunterladen (es sind allerdings mindestens 24~Megabytes) oder man
bestellt sich dort "`MiK\TeX{} on \versalien{CD-R}"'\@.

In beiden F�llen muss man das Setup-Programm aufrufen.  Wenn man von \CD{}
installiert, sollte man mindestens die "`Large"'-Installation w�hlen, ansonsten
reicht die "`Small"'-Variante.  Das ganze dauert je nach Fall zwischen f�nf und
f�nfzehn Minuten.


\subsection{FreePDF}
\label{sec:FreePDF}
\index{FreePDF}
\index{PDF}
\index{Ghostscript}
\index{Postscript}

Das Programm FreePDF ist nicht unbedingt f�r ein funktionierendes \LaTeX-System
n�tig.  Es erlaubt, aus jeder beliebigen Windows-Anwendung heraus \PDF-Dateien
zu erzeugen, die man als Bilder in \LaTeX{}-Texte einbinden kann.  Wir
empfehlen, es immer zu installieren, wenn man Abbildungen aus beispielsweise
Corel Draw, PowerPoint, Excel oder Origin in seinen Texten verwenden m�chte.

Es ist kostenlos im Internet unter `\url{freepdfxp.de/fpxp.htm}' zu beziehen.
Dort klickt man auf "`Download"' und mu� \emph{zwei} Dateien herunterladen und
installieren:
\begin{enumerate}
\item "`GhostScript"' ($10$~MByte) und
\item "`FreePDF~XP"' selber ($0{,}5$~MByte).
\end{enumerate}


\subsection{Der Editor}
\index{Editor}
\index{TeXnicCenter@\TeXnicCenter}
\index{ToolsCenter}

F�r den Benutzer ist der wichtigste Teil von \LaTeX{} der \emph{Editor}.  Im
Editor gibt man seinen Text ein, schaut ihn sich in der Schnellvorschau an und
druckt ihn aus.  In diesem Tutorium verwenden wir einen Editor namens
\TeXnicCenter.  Es gibt Alternativen, aber der \TeXnicCenter{} ist leicht zu
handhaben und bietet alles, was man braucht.  Seine Webseite ist
`\url{www.toolscenter.org}'.

Man bekommt ihn direkt aus dem Internet unter
\begin{quote}
  \url{http://www.texniccenter.org/front_content.php?idcat=50}
\end{quote}
Man muss dort die oberste Datei ausw�hlen (auf "`SourceForge.net"' klicken,
$5{,}5$\,MByte) und auf dem eigenen Rechner starten.  W�hrend der Installation
best�tigt man alle Vorgaben.  \textebf{Achtung:} Der \TeXnicCenter{} sollte
stets als allerletztes installiert werden!


%%% Local Variables: 
%%% mode: latex
%%% TeX-master: "latex-tutorium"
%%% End: 
