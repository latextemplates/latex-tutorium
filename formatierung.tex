%       $Id: formatierung.tex,v 1.19 2005/06/28 18:30:35 bronger Exp $    
%
%     formatierung.tex -- Part of the LaTeX Tutorium
%     Copyright 2004 Project Members of
%                    http://sourceforge.net/projects/latex-tutorium/
%                    
%
%   This program is free software; you can redistribute it and/or
%   modify it under the terms of the Artistic License 2.0 as published
%   by Larry Wall.  You should have received a copy of the Artistic
%   License 2.0 along with this program in the file COPYING; if not,
%   you can get it at
%     http://dev.perl.org/rfc/346.html
%   or contact the current maintainers of the LaTeX Tutorium.
%
%   This program is distributed in the hope that it will be useful, but
%   WITHOUT ANY WARRANTY; without even the implied warranty of
%   MERCHANTABILITY or FITNESS FOR A PARTICULAR PURPOSE.  See the
%   Artistic License 2.0 for more details.
%
%   This file may only be distributed together with a copy of the LaTeX
%   Turorium.
%
%   The LaTeX Tutorium consists of all files listed in manifest.txt.


\section{Formatierung}

\subsection{Fett, kursiv etc}

\begin{table}[htbp]
  \caption{Einige M�glichkeiten der Textformatierung}
  \label{tab:Auszeichnungen}
  \vspace{1ex}
  \centering
  \begin{tabular}{@{}lcc@{}}
    \toprule
    Formatierung              &  Tastenk�rzel        &  Beispiel                   \\
    \midrule
    \emph{kursiv}             &  \Ctrl+\keystroke E  &  \lstinline|\emph{Wort}|    \\
    \textbf{fett}             &                      &  \lstinline|\textbf{Wort}|  \\
    \textsc{Kapit�lchen}      &                      &  \lstinline|\textsc{Wort}|  \\
    \textsf{serifenlos}       &                      &  \lstinline|\textsf{Wort}|  \\
    \texttt{Schreibmaschine}  &                      &  \lstinline|\texttt{Wort}|  \\
    \bottomrule
  \end{tabular}
\end{table}

Wenn man ein Wort oder einen Satz betonen m�chte, druckt man ihn am besten
\emph{kursiv}.  Das geht mit \Ctrl+\mbox{\keystroke E,} welches den Befehl
\lstinline{\emph} (engl.~\emph{emphasize}--\emph{betonen}) einf�gt:
\begin{lstlisting}
... druckt man ihn am besten \emph{kursiv}.
\end{lstlisting}
In den allermeisten Texten kommt man mit Kursivdruck aus.  Trotzdem sind in
Tabelle~\vref{tab:Auszeichnungen} noch weitere M�glichkeiten aufgelistet.  Man
kann sie auch kombinieren:
\begin{lstlisting}
\textbf{\emph{fett-kursiv}}
\end{lstlisting}


\begin{table}[t]
  \caption{Einige Schriftarten-Befehle.}
  \label{tab:Schriftarten}
  \vspace{1ex}
  \newcommand{\tone}{\fontencoding{OT1}}
  \centering\small
  \begin{tabular}{@{}lll@{}}
    \toprule
    Befehl & Schriftart & wirkt auf\\
    \midrule
    \verb|\usepackage{mathptmx}| & \fontfamily{ptm}\selectfont Times New Roman & Normaltext \& Formeln \\
    \verb|\usepackage{mathpazo}| & \fontfamily{ppl}\selectfont Palatino        & Normaltext \& Formeln \\
    \verb|\usepackage{courier}|  & \fontfamily{pcr}\selectfont Courier         & Text in Schreib-\\
                                 &                 & \quad maschinenschrift \\
    \multicolumn{2}{@{}l}{\texttt{\textbackslash usepackage[scaled]\char'173 helvet\char'175}} & \\
    & \fontfamily{phv}\selectfont Helvetica & Serifenloser Text      \\
    \verb|\usepackage{bookman}|  & \fontfamily{pbk}\selectfont Bookman         & Normaltext             \\
    \verb|\usepackage{newcent}|  & \fontfamily{pnc}\selectfont New Century     & Normaltext             \\
                                 & \quad\fontfamily{pnc}\selectfont Schoolbook&                        \\
    \verb|\usepackage{avant}|    & \fontfamily{pag}\selectfont Avant Garde     & serifenlosen Text      \\
    \verb|\usepackage{charter}|  & \fontfamily{bch}\selectfont Charter         & Normaltext             \\
    \verb|\usepackage{chancery}| & \fontfamily{pzc}\selectfont Zapf Chancery   & Normaltext             \\
    \emph{(Voreinstellung)}      & \fontfamily{cmr}\tone\selectfont Computer Modern & Normaltext \& Formeln  \\
    \emph{(Voreinstellung)}      & \fontfamily{cmss}\tone\selectfont CM Sans Serif  & serifenlosen Text      \\
    \emph{(Voreinstellung)}      & \fontfamily{cmtt}\tone\selectfont CM Typewriter  & Text in Schreib-       \\
                                 &                 & \quad maschinenschrift \\
    \bottomrule
  \end{tabular}
\end{table}

\subsection{Schriftarten}

Es ist m�glich, nahezu jede beliebige Schriftart, die man besitzt, mit \LaTeX{}
zu benutzen.  Da \LaTeX{} sehr viele Informationen �ber Schriftarten ben�tigt,
ist das aber viel Arbeit.  Gl�cklicherweise haben sich bereits andere diese
M�he gemacht, zumindest f�r die interessanten Schriftarten.

Die Beispiele in diesem Tutorium werden in "`Times New Roman"' gedruckt, die
Schriftart, die meist mit Word benutzt wird.  Wenn man sich nochmal das
Muster-Dokument auf Seite~\pageref{Minimaldokument} anschaut, sieht man dort in
den Zeilen~\ref{lst:mathptmx-courier} und~\ref{lst:helvet}
\begin{lstlisting}
\usepackage{mathptmx,courier}
\usepackage[scaled]{helvet}
\end{lstlisting}
Diese kryptischen Befehle stellen auf Times (f�r normaler Text \& Formeln),
Courier (f�r Schreibmaschinen-Schrift) und Helvetica (f�r serifenlos).

Will man andere Schriftarten haben, muss man diese Zeilen l�schen und einen
oder mehrere der in Tabelle~\vref{tab:Schriftarten} genannten Schriftartbefehle
an dieser Stelle einf�gen.


\subsection{Besondere Textsymbole, Trennungen und Leerr�ume}
\label{sec:sonderzeichen}

Einige Zeichen haben in \LaTeX{} eine besondere Bedeutung.  Wir hatten
beispielsweise schon gesehen, dass das Prozent-Zeichen bewirkt, dass der Rest
der Zeile ignoriert wird.  Was aber, wenn man wirklich ein Prozent-Zeichen
eingeben m�chte?

In diesem Fall muss man einfach einen Backslash~``\verb|\|'' davorstellen.  Man
kann also schreiben
\begin{lstlisting}
Die SPD kam auf 33,3\%.
\end{lstlisting}
Das gilt ebenso f�r die Zeichen \&, \# und~\$.

Zwei Bindestriche ``\verb|--|'' f�gen einen Gedankenstrich ein -- den man auch
f�r von-bis-Ausdr�cke verwenden kann: Von 14--15~Uhr.  Achtung, das ist kein
Bindestrich!  Der ist weiterhin blo� ein einzelnes~``\verb|-|''.

Die "`G�nsef��chen"' m�ssen in \LaTeX{} etwas seltsam eingegeben werden:
\begin{lstlisting}
Ein sogenannter "`Roter Riese"' ist ein
Stern, der ...
\end{lstlisting}
allerdings erledigt das der Editor bereits automatisch.

\begin{table}
  \caption{Besondere Zeichen in \LaTeX{}}
  \label{tab:Sonderzeichen}
  \vspace{1ex}
  \centering
  \begin{tabular}{@{}lcl@{}}
    \toprule
    \LaTeX-Befehl  &  Aussehen  &  Beschreibung \\
    \midrule
    \verb|\&|, \verb|\$|, \verb|\%|, \verb|\#| & \&, \$, \%, \# & \\
    \verb|\S{}|  & \S{}  &  Paragraphen-Zeichen \\
    \verb|--|  &  --  &  Gedankenstrich, von--bis-Strich \\
    \verb|\dots{}|  &  \dots  &  Auslassungspunkte \\
    \verb|"`|\dots\verb|"'|  &  "`\dots"'  &  G�nsef��chen \\
    \verb|~|   &      &  Leerschritt, Zeilenumbruch verboten \\
    \verb|\,|   &     &  halber Leerschritt, Zeilenumbruch \\
                &          &  \quad verboten \\
    \verb|"~|  &  -   &  Bindestrich, Zeilenumbruch verboten \\
%    \verb|""|  &  (\emph{nichts})    &  Zeilenumbruch erlaubt, keinen \\
%               &                     &  \quad Bindestrich einf�gen \\
    \verb|\LaTeX{}|  &  \LaTeX{}  &  offizielles \LaTeX-Logo \\
    \bottomrule
  \end{tabular}
\end{table}
Ach ja, last but not least: "`\LaTeX{}"' schreibt man "`\lstinline|\LaTeX{}|"'.

\bigskip\noindent
Manchmal will man die Silbentrennung oder den Zeilenumbruch beeinflussen.  Da
bietet \LaTeX{} einfache und praktische M�glichkeiten.  Eine Tilde ``\verb|~|''
beispielsweise erzeugt ein Leerzeichen, an dem nie ein Zeilenumbruch
stattfindet:
\begin{lstlisting}
Die Pr�fung leitet Prof.~Dr.~Obermeyer.
\end{lstlisting}
Dasselbe verursacht ein ``\verb|\,|'', allerdings ist das nur ein halbes
Leerzeichen:
\begin{lstlisting}
Er war 175\,cm gro�.
\end{lstlisting}
Wenn ich einen Bindestrich brauche, an dem nie die Zeile umbrochen werden darf,
gebe ich ihn mit \verb|"~| ein:
\begin{lstlisting}
Seine Aussage im O"~Ton war, dass ...
Das gibt Abz�ge in der B"~Note.
\end{lstlisting}
So kann man garantieren, dass solche F�lle immer richtig getrennt werden.
Tabelle~\vref{tab:Sonderzeichen} enth�lt eine �bersicht �ber diese Befehle.

Falls einzelne W�rter falsch getrennt werden sollten, f�hrt man einfach alle
m�glichen Trennungen des Wortes in der Pr�ambel mit Hilfe des Befehls
\lstinline|\hyphenation| auf:
\begin{lstlisting}
\hyphenation{Text-ein-ga-be Zei-len-um-bruch}
\end{lstlisting}

%%% Local Variables: 
%%% mode: latex
%%% TeX-master: "latex-tutorium"
%%% End: 
