%       $Id: gifs-etc.tex,v 1.10 2005/03/20 21:55:48 bronger Exp $    
%
%     gifs-etc.tex -- Part of the LaTeX Tutorium
%     Copyright 2004 Project Members of
%                    http://sourceforge.net/projects/latex-tutorium/
%                    
%
%   This program is free software; you can redistribute it and/or
%   modify it under the terms of the Artistic License 2.0 as published
%   by Larry Wall.  You should have received a copy of the Artistic
%   License 2.0 along with this program in the file COPYING; if not,
%   you can get it at
%     http://dev.perl.org/rfc/346.html
%   or contact the current maintainers of the LaTeX Tutorium.
%
%   This program is distributed in the hope that it will be useful, but
%   WITHOUT ANY WARRANTY; without even the implied warranty of
%   MERCHANTABILITY or FITNESS FOR A PARTICULAR PURPOSE.  See the
%   Artistic License 2.0 for more details.
%
%   This file may only be distributed together with a copy of the LaTeX
%   Turorium.
%
%   The LaTeX Tutorium consists of all files listed in manifest.txt.


\section{Die zwei Arten von Grafiken}
\label{sec:epsgif}

Es gibt zwei Arten von Grafiken:
\begin{description}
\item[Vektorbilder] sind Diagramme oder Strichzeichnungen.  Sie werden gemalt
  mit beispielsweise Corel Draw, Adobe Illustrator oder dem Zeichnungs-Modul
  von OpenOffice.org.  Andere Programme wie Origin oder Excel erlauben es
  ebenso, die Ergebnisse als Vektorbild auszugeben.

  Solche Bilder sollten immer als \PDF-Datei gespeichert werden und so in dan
  \LaTeX-Text eingebunden werden.
  
\item[Bitmaps] sind im Allgemeinen Fotos, z.\,B. von einer Digitalkamera oder
  einem Scanner.  Leider werden Bitmaps auch benutzt, wenn man von deinem
  Digramm aus irgendeinem Grund kein Vektorbild hat.  Das ist nicht fatal, aber
  man sollte sich dar�ber im Klaren sein, dass es alles andere als optimal ist.
  
  Es gibt sehr viele Varianten f�r Bitmaps, z.\,B. \JPEG, \PNG, \GIF, \TIFF{}
  und \BMP\@.
\end{description}
Wie \PDF{}s, \JPEG{}s und \PNG{}s eingebunden werden, haben wir schon auf
Seite~\pageref{sec:Grafiken} beschrieben.  Das reicht meistens, und damit kommt
\LaTeX{} direkt zurecht.  Alle anderen Grafik-Formate m�ssen irgendwie in diese
Formate umgewandelt werden.


%%% Local Variables: 
%%% mode: latex
%%% TeX-master: "latex-tutorium"
%%% End: 
